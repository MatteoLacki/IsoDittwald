\documentclass[runningheads,a4paper]{llncs}
%\documentclass[letter,11pt]{article}
%geometry -> one inch marge.
\usepackage{./sty/mystyle}

\usepackage{url}
\urldef{\mailsa}\path|mateusz.lacki@biol.uw.edu.pl|
\urldef{\mailsb}\path|aniag@mimuw.edu.pl|

\newcommand{\keywords}[1]{\par\addvspace\baselineskip
\noindent\keywordname\enspace\ignorespaces#1}

\begin{document}
\mainmatter  

\title{Law of Localised Fine Structure}

\subtitle{\textit{with application in mass spectrometry}\thanks{
	This research was partially supported by Polish National Science Center grant $\text{n}^\text{o}$ 2011/01/B/NZ2/00864.
}}

% \subtitle{\textit{with application in mass spectrometry proteomic studies.}}

\titlerunning{Mass Spec Fine Structure Distribution}

\author{Mateusz Krzysztof \L\k{a}cki
\and Anna Gambin}

\authorrunning{Fine Structure Distribution in Mass Spectrometry}


% the affiliations are given next; don't give your e-mail address
% unless you accept that it will be published
\institute{
  Faculty of Mathematics, Informatics and Mechanics\\ University of Warsaw, Banacha 2, 02-097 Warszawa, Poland\\
  \mailsa\\
  \mailsb
  % \url{http://bioputer.mimuw.edu.pl}
}

% NB: a more complex sample for affiliations and the mapping to the
% corresponding authors can be found in the file "llncs.dem"
% (search for the string "\mainmatter" where a contribution starts).
% "llncs.dem" accompanies the document class "llncs.cls".

\toctitle{Mass Spec Fine Structure Distribution}
\tocauthor{Mateusz \L\k{a}cki}
\maketitle

\begin{abstract}
	This paper presents a brand new methodology to deal with isotopic fine structure calculations. By using the Poisson approximation in an entirely novel way, we introduce mathematical elegance into the discussion on the trade-off between resolution and tractability. Our considerations unify the concepts of fine-structure, equatransneutronic configurations, and aggregate isotopic structure in a natural and simple way. We show how to boost the theoretical resolution in a seemingly costless way by several orders of magnitude with respect to the already very efficient algorithms operating on isotopic aggregates. We also develop an effective new way to obtain the important peaks in the most disaggregated isotopic structure localised in a precise region in the mass domain.
\keywords{Isotopic Fine Structure, Poisson Approximation, Stable Isotopes, Avergine Model.}
\end{abstract}

% \listoftodos

%!TEX root = ../DeGaulle.tex
\section{Introduction}

There are many reasons why mass spectrometry analysis is hard. It is hard in that there are potentially many many sources of interferences that can distort the information about the actual composition of a sample. The study of the nature of these interferences is needed to achieve the goal of making out of mass spectrometers yet more reliable an identification tool. 

Part of the noise in the mass to charge domain is innately related to the elements themeselves and stems from the existence of isotopes. It is because of them that a given analyte is represented as a series of peaks, a spectrum, rather that only one peak. The theoretical underpinnings of how to mathematically model the impact of isotopes are already well established, see \cite{Valkenborg2012Isotopic}. The main idea behind the model is to abstract from the exact positionings of the extra neutrons on a particular chemical compound and thus concentrate only on their relative amounts among all atoms of a given element. Assuming that the isotopic configurations are independent and follow the element dependent distribution, one arrives to the conclusion  that the correct law describing occurence of different isotopes in a chemical compound is the product of multinomial distributions. 

There is one huge problem with that law: together with the growth of molecule one observes an exponential growth in the number of possibile isotope configurations, which precludes their direct enumeration. To solve this problem, different semplifications were proposed, amounting to different ways of binning configurations together explicitly \cite{Claesen2012Efficient}, by hiding them under the guise of Fourier Transform \cite{Rockwood1995Relationship}, or by \dots 

\todo[inline]{However, it is considered of paramount importance in Mass Spectrometry to develop machines with still higher resolution powers and it is very likely that this trend will continue. Even today there are machines that already can distinguish peaks attributed to different configurations with the same number of extra neutrons.  }
\todo[inline]{Add Olson and others but Olson above all.}

Here we propose to approach the problem of fine structure so that it overcomes the shortcomings of the aggregate model, as used in \cite{Claesen2012Efficient}. That particular model bins together configurations having the same number of extra neutrons distributed on different atoms. For instance, if one considers water molecule \ce{H_2 O}, the model would glue together configuration with one extra neutron only on the first hydrogen togeter with that having it on the second together with that on oxygen atom. We devise an algorithm to deaggregate these probability clusters. We call the peaks obtained via that algorithm a {\it localised fine structure}. 


What motivates the solution to this problem is a search for better molecule fingerprints. The development of new mass spectrometers capable of distinguishing differences in masses of neutrons is proceeding at a vigorous pace. Soon, scientists will face the need of more detailed models than those abstracting from mass defects. It is also common for chemists to search for presence of specific substance in the sample. Usually this is done by looking at some highly specific range in the mass domain of the gathered spectra. Our model provides deeper insight to what might happen while focussing on that particular bit of collected data: conditioning on configurations with the same number of extra neutrons translates directly into focussing in a specific region of the mass-to-charge domain.


The algorithm assumes that one can easily find a peak not far from the most probable one and that the distribution is close to what one would call unimodal\footnote{We provide a precise definition of unimodality for discrete probability distributions in Section \dots} and that the most of distributions probability lies in a rather small neighbourhood of the mode. Both the guess about the starting point and the way the neighbourhood gets explored depend on the Poisson approximation to the distribution under study. To our best knowledge this type of approximation have not yet been used for algorithmic purposes. It has been used however in the context of proteomic and peptide research: in \cite{Breen2000AutomaticPeak} it is being used for high throughput protein identification; its use was revalidated in \cite{Valkenborg2007UsingPoisson} in case of peptides.


We also observe that the use of Poisson approximation gives a theoretical explanation for the equatransneutronic binning used in \cite{Olson2009Calculations} and actually helps deaggregating results obtained using that approach as well.

\todo[inline]{ Diophantine equations. }

% For instance, by neglecting the fact that the mass of additional neutron with respect to the lightest isotope can discriminate between elements, one arrives at a probability distribution that is computationally tractable in different ways: either algebraically \cite{Claesen2012Efficient}, by use of Fourier Transform \cite{Rockwood1995Relationship}, dynamic programming , Li2008HierarchicalAlgorithm  


% The main idea is to treat a chemical compound, such as \molecule, as a set of 

% The problem is that usually a more detailed approach 

% There is usually much of indeterminacy in the 
% Part of the problem lies in the nature of chemical sound that prohibits us from getting to know the exact composition of an analyte under study. One must acknowledge also the fact, that part of indeterminacy in the obtained results stems from the existence of isotopes of different elements. Mass 

% A result of a mass spectrometry experiment is usually a complicated spectrum 

% The concept of isotopic structure is crucial for the proper interpretation of mass spectrometry spectra. 

% The fact that the result of a mass spec analysis is usually an entire spectrum in the mass to charge domain rather than an individual peak can to some extent be attributed to the existance of the isotopic structure. 

% Part of the indeterminacy in the mass to charge domain can be attributed to the existence of isotopes.

% In this article we are asking ourselves a question of how to 

% The interpretation of a mass spectrometry analysis is rarely trivial, the number of possible sources of  interferences being a very large one. Even a perfectly carried out experiment usually  

 % of a given analyte rarely does appear in mass spec spectra in form of a single peak and a lot of the indeterminacy in weight can be attributed to the existence of isotopes.  


	% Cite all others.



% One of the first uses of Poisson approximation in the context of isotopic structure calculations is to be found in \cite{Breen2000AutomaticPeak}, where it is being used for high throughput protein identification. The use of approximation was validated in \cite{Valkenborg2007UsingPoisson} in case of peptides. Authors of this article also raise the important topic of approximations validity, from a purely applied viewpoint. 





% All interesting citations \cite{Claesen2012Efficient} and \cite{Valkenborg2012Isotopic} and \cite{Olson2009Calculations} and \cite{Ipsen2014Efficient} and \cite{Valkenborg2008ModelBased} and \cite{Valkenborg2007UsingPoisson} and \cite{Rockwood1995Relationship} and \cite{Breen2000AutomaticPeak}.
%!TEX root = ../DeGaulle.tex
\section{Approximations}

By an isotopic configuration we understand information on numbers of different isotopes a chemical compound in the sample is made of. For the purpose of simplicity, we focus here on chemical compounds composed of carbon, hydrogen, nitrogen, oxygen, and sulfur; still, results of this section generalize to any compound whatsoever. Thus, we concentrate on compounds like \molecule, where the low case letters describe the numbers of atoms of particular element type. Among such compounds one can already find peptides and proteins. An isotopic configuration could be represented by an extended empirical formula, 
\begin{equation}\label{long chemical formula}
	\text{\moleculeIsotopic}.
\end{equation}

In the above representation, small letters with indices represent counts of different atoms with indices displaying the number of additional neutrons an isotope has with respect to the lighest possible isotopic variant. 

Rather than \eqref{long chemical formula}, we shall be using an equivalent probabilistic notation, treating upper case letters, like \ce{^{12}C}, as random variables and considering small case letters, $\cem{c_0}$, to be their realisations. An expression like $A = \{ \ce{^{13}C} = \cem{c_1},\, \ce{^{2}H} = \cem{h_1} \}$ is shorthand for saying: let us focus on all configurations \eqref{long chemical formula} that have \ce{c_1} heavy carbons and \ce{h_1} deuters in total.


Following \cite{Kienitz1961MassSpectrometry}, one assumes that the law of vector
\begin{equation}\label{long chemical vector}
	\left( \cem{^{12}C},\, \cem{^{13}C},\, \cem{^{1}H},\, \cem{^{2}H},\, \cem{^{14}N},\, \cem{^{15}N},\, \cem{^{16}O},\, \cem{^{17}O},\, \cem{^{18}O},\, \cem{^{32}S},\, \cem{^{33}S},\, \cem{^{34}S},\, \cem{^{36}S} \right),	
\end{equation}
given \molecule, is a product of independent multinomial distributions,
{\small\begin{equation}\label{product of multinomials}
	\MM = \mathrm{Multi} \Big( \prob(\cem{^{12}C}), \prob(\cem{^{13}C}); c \Big)
	\otimes \dots \otimes 
	\mathrm{Multi} \Big( \prob(\cem{^{32}S}), \prob(\cem{^{33}S}), \prob(\cem{^{34}S}), \prob(\cem{^{36}S}); s \Big),	
\end{equation}}
where the probabilities of observing particular isotopes, $\prob(\cem{^{12}C})$, \dots, $\prob(\cem{^{36}S})$, are established in independent experiments, cf. Table \ref{basic info on isotopes table}. For instance, the probability of a given carbons configuration $(\cem{c_0}, \cem{c_1})$ equals
% \begin{equation*}
% 	\mathrm{Multi} \left( \prob(\cem{^{12}C}), \prob(\cem{^{13}C}); c \right)
% 		\Big( (\cem{c_0}, \cem{c_1}) \Big) = 
% 	\begin{pmatrix}
% 		\cem{c} \cr \cem{c_0}, \cem{c_1}  
% 	\end{pmatrix} \prob(\cem{^{12}C})^\cem{c_0} \prob(\cem{^{13}C})^\cem{c_1}
% \end{equation*}
$$
	\mathrm{Multi} \left( \prob(\cem{^{12}C}), \prob(\cem{^{13}C}); c \right)
		\Big( (\cem{c_0}, \cem{c_1}) \Big) = 
	\begin{pmatrix}
		\cem{c} \cr \cem{c_0}, \cem{c_1}  
	\end{pmatrix} \prob(\cem{^{12}C})^\cem{c_0} \prob(\cem{^{13}C})^\cem{c_1}
$$
and it should be multiplied by similar expression for hydrogen, nitrogen, oxygen and sulfur to obtain probability for configuration \eqref{long chemical formula}.


Observe, that given \molecule, part of the information in representation \eqref{long chemical vector} is redundant and can be shortened by neglecting counts of the lightest isotope variants, leaving us with 
\begin{equation}\label{short chemical vector}
 	\left( \cem{^{13}C},\, \cem{^{2}H},\, \cem{^{15}N},\, \cem{^{17}O},\, \cem{^{18}O},\, \cem{^{33}S},\, \cem{^{34}S},\, \cem{^{36}S} \right).	
\end{equation}
Missing terms can be retrieved from relationships $\cem{^{12}C} + \cem{^{13}C} = \cem{c}$, $\cem{^{1}H} + \cem{^{2}H} = \cem{h}$, and so on, that occur with probability one.

\begin{mydef}\label{localised fine structure definition}
	We call the set of configurations  
	{\small
		\begin{equation}\label{LFS_K}
			LFS_K	=
			\left\{ 
				\cem{^{13}C} + \cem{^{2}H} +  \cem{^{15}N} +  \cem{^{17}O} +  \cem{2 $\times$^{18}O} +  \cem{^{33}S} +  \cem{2 $\times$^{34}S} + \cem{4 $\times$^{36}S} = K	
			\right\}
		\end{equation}
	}
	a \emph{\textbf{localised fine structure} with $K$ extra neutrons}.  	
\end{mydef}

The reason for numbers 2 and 4 appearing above is that \ce{^{18}O} and \ce{^{34}S} have two additional neutrons, and \ce{^{36}S} -- four; cf. Table \ref{basic info on isotopes table}.

The problem of enumerating all elements of $LFS_K$ is known as the money exchange problem. In general, it corresponds to finding all integer solutions $(x_1, \dots, x_k)$ of a {\it Linear Diophantine Equation}  
\begin{equation}\label{Linear Diophantine Equation}
	d_1 x_1 + \dots + d_k x_k = K,
\end{equation}
where $(d_1, \dots, d_k)$ are integer coefficients. According to \cite{Agnarsson2002OnTheSylvesterDenumerants}, if the greatest common divisor of $(d_1, \dots, d_k)$ is equal to one, then the number of solutions to Eq. \eqref{Linear Diophantine Equation} is approximately $\frac{K^{k-1}}{(k-1)! d_1 \dots d_k}$. Carbon has only one additional isotope, so $\exists_i d_i = 1$ in \eqref{Linear Diophantine Equation}. The above estimate encompasses therefore all of organic chemistry and proteomics. 

Nonetheless, since configurations in $LFS_K$ are naturally prioritized by probability \eqref{product of multinomials} one would be satisfied with enumerating only the most probable ones. 

%  is simply a subset of all possible configurations \eqref{long chemical formula} with additional constraint 
% \begin{equation}\label{Simple Diophantine Equation}
% 	\cem{c_1} + \cem{h_1} + \cem{n_1} + \cem{o_1} + 2 \cem{o_2} + \cem{s_1} + 2 \cem{s_2} + 4 \cem{s_4} = K.
% \end{equation}

\begin{Problem}\label{Problem of finding LFS_K configurations.}
	For a given $K$, find a small set $B \subset LFS_K$ of configurations s.t. 
	\begin{equation}\label{problem equation}
		\MK (B) := \frac{ \MM(B) }{ \MM( LFS_K ) } \approx 1\,,	
	\end{equation} 
	where $\MK$ is the product of multinomial laws \eqref{product of multinomials} conditioned on the set of configurations in $LFS_K$ and is referred to as \emph{\textbf{The Law of Localised Fine Structure}}.
\end{Problem}


In statistical terms, we are interested in approximating some critical set of large probability, as measured by the {\it Law of Localised Fine Structure}. 


Why should one study law described by \eqref{problem equation} in the first place? Simply because the masses of different configurations in $LFS_K$ concentrate around the compound's monoisotopic mass shifted to the right by $K$ Daltons; c.f \cite{Hughey2001KendrickMassDefect}. For medium sized compounds, $LFS_K$'s for different $K$ should in principle form disjoint clusters in the mass to charge domain, with some interference for bigger compounds. Studying $LFS_K$ guarantees exploration of a precise region in the mass to charge domain.


To solve Problem \ref{Problem of finding LFS_K configurations.} we approximate measure $\MK$ by a more analytically tractable measure $\QK$ defined on the $LFS_K$. We then devise an algorithm to find a possibly small set of configurations $B^* \subset LFS_K$, s.t. $\QK (B^*) \approx 1$. Since $\QK \approx \MK$, so $\MK (B^*) \approx 1$ and $B^*$ solves Problem \ref{Problem of finding LFS_K configurations.}, possibly suboptimally.


A natural way to define proper $\QK$ is to first approximate $\MM$ by some $\QQ$ and then pose $\QK (\circ) := \frac{\QQ (\circ \cap LFS_K) }{\QQ(LFS_K)}$, i.e. condition $\QQ$ on the occurrence of configurations from $LFS_K$. To prove it works, we have to first mention, that by approximation we understand convergence in distribution, as described in \cite{Kallenberg2002FoundationsOfModernProbability}. Then, we make use of the following lemma: 

\begin{lemma}\label{conditional convergence lemma}
	Let $\mu^{[n]}, \mu$ be discrete measures. If $\mu^{[n]}$ converges in distribution to $\mu$, $\mu^{[n]} \rightharpoonup  \mu$, and an event $A$ has nonzero probability under any of that measures, $\underset{n}{\forall} \mu^{[n]}(A)\,,\, \mu(A) > 0$, then measures conditioned by $A$, $\mu^{[n]}_A (\circ) := \frac{\mu^{[n]} ( \circ \cap A)}{\mu^{[n]}(A)}$ converge in distribution to $\mu_A (\circ) := \frac{ \mu( \circ \cap A) }{ \mu(A) }$; or $\mu^{[n]}_A \rightharpoonup \mu_A$ for short.
\end{lemma}  
Proof is to be found in \textbf{Appendix}.  


Let us now unveil the usefulness of Lemma \ref{conditional convergence lemma}. There is an entire family of measures mentioned in it, $\mu^{[n]}$. We assume, that one of them is simply our initial measure: there exists $n^*$ s.t. $\MM = \mu^{[n^*]}$. Also, we assume the approximation of $\mu^{[n^*]}$ by measure $\mu$ is already o good one. Our choice for $\mu$ is to be the product of independent Poisson measures, which is stimulated by the following, well known lemma.


\begin{lemma}\label{weak convergence of multinomial to Poissons lemma}
	If all\,\,$\lim_{n\to \infty} n p_{k,n}= \lambda_k$ exist for $k \in \{1,\dots, w\}$, then 
	
	\begin{equation}\label{weak convergence of multionial to Poissons equation}
		\mathrm{Multi}\left( p_1^{[n]}, \dots, p_w^{[n]}; n \right) 
			\rightharpoonup 
		\mathrm{Poiss}( \lambda_1) \otimes \dots \otimes \mathrm{Poiss}( \lambda_w ),	
	\end{equation}
	where $\mathrm{Poiss}$ stands for the Poisson distribution, $\mathrm{Poiss}(\lambda)(k) 	= \frac{\lambda^k}{k!}e^{-\lambda}$.
	
\end{lemma}


In Lemma \ref{weak convergence of multinomial to Poissons lemma} one assumes that the number of trials $n$ goes to infinity. In our model this corresponds to an infinite enlargement of the compound. The existence of limits assumes that this enlargement is done so that on such an idealized compound only the lightest isotopes would appear infinitely often. Moreover, since the support of any Poisson distribution is equal to the set of all integer numbers, the state space of configurations gets significantly enlarged and contains configurations that are nonphysical for any real chemical compound. For instance, positive probabilities would be prescribed to configurations with numbers of isotopes greater then the number of possible places for them on any finite compound. Observe also, that the probabilities $p_k^{[n]}$ are pending towards zero: for good approximation one would expect therefore the probabilities of observing heavier isotopes, e.g. quantities like $\prob(\cem{^{13}C}), \prob(\cem{^{2}H}), \dots, \prob(\cem{^{36}S})$, to be relatively small. That is the case -- cf. Table \ref{basic info on isotopes table}.
% \footnote{There is no mathematical incongruence here, however, since the approximation assumes that the limiting compound is of infinite size. For mathematical correctness we also note, that we can transfer virtually any initial measure $\MM$ on the enlarged state space, i.e. where the approximation is defined, simply by assuming, that $\MM$ measure on any nonphysical configuration equals zero.}. 


Observe, that Lemma \ref{weak convergence of multinomial to Poissons lemma} defines a proper limit for just one multinomial distribution, whereas $\MM$ is a product thereof. The problem is other than what to do with products: one can approximate independently each multinomial. However, the quality of such approximation depends on all the counts of different elements in a molecule. For instance, in case of \molecule\,the better the approximation\footnote{The {\it goodness} of approximation is expressed in the total variance distance; see \cite{Roos1999OnTheRateOfMultivariatePoissonConvergence}.} the bigger the smallest among numbers $(\cem{c}, \cem{h}, \cem{n}, \cem{o}, \cem{s})$. Due to the polymer structure, one would expect some more information could be revealed on that matter for proteins and peptides. Indeed, empirical research by Senko et al. \cite{Senko1995Determination} established the concept of $m$-avergine, i.e. an averaged protein: any protein composed of $m$ amino acids should have its mass approximately equal to the mass of the idealised compound 
\begin{equation*}
	\cem{C}_{\lfloor m \times 4.9384\rfloor} 
	\cem{H}_{\lfloor m \times 7.7583\rfloor} 
	\cem{O}_{\lfloor m \times 1.4773\rfloor} 	
	\cem{N}_{\lfloor m \times 1.3577\rfloor} 
	\cem{S}_{\lfloor m \times 0.0417\rfloor}.
\end{equation*}

The weakest link in the approximation might result from small numbers of sulfur. This is an acknowledged problem in empirical studies, as exposed in \cite{Valkenborg2007UsingPoisson}. The longer the polymers however, the smaller the differences should be. 


The final questions is: what values should be used as $\lambda$'s in Lemma \ref{weak convergence of multinomial to Poissons lemma}? We {\it calibrate} those values by equating them to the averages of the multinomial distributions from \eqref{product of multinomials}: in case of carbon we set $\lambda_\cem{^{13}C} \approx \cem{c} \times \prob( \cem{^{13}C} )$.  In contrast to our method, $\lambda$'s in \cite{Breen2000AutomaticPeak,Valkenborg2007UsingPoisson} are chosen to be the minimizers in a free parameter optimisation scheme with $\chi^2$ penalty\footnote{Note however, that these two solutions should not differ too much for larger compounds, for it is known that both the Poisson and Multinomial distributions are concentrated near their means, see \cite{Bobkov1998OnModifiedLogarithmicSobolev}.}. 


All in all, the probability assigned to event
\begin{equation*}
 	\left\{ \cem{^{13}C} = \cem{c_1},\, \cem{^{2}H} = \cem{h_1},\, \cem{^{15}N} = \cem{n_1},\, \cem{^{17}O} = \cem{o_1},\, \cem{^{18}O} = \cem{o_2},\, \cem{^{33}S} = \cem{s_1},\, \cem{^{34}S} = \cem{s_2},\, \cem{^{36}S}= \cem{s_4} \right\}	
\end{equation*} 
is given by
\begin{equation}\label{QK Nominator}
	\poiss{\text{c}}{13}{1}
	\poiss{\text{h}}{2}{1}
	\poiss{\text{n}}{15}{1}
	\poiss{\text{o}}{17}{1}
	\poiss{\text{s}}{33}{1}
		e^{ - \mu}
	\poiss{\text{o}}{18}{2}	
	\poiss{\text{s}}{34}{2}
		e^{ - \eta }		
	\poiss{\text{s}}{36}{1}
		e^{ - \gamma },
\end{equation}
where 
\begin{align*}\label{intensities summed equation}
	\mu 	&=	\lambda_\cem{^{13}C} + \lambda_\cem{^{2}H} + \lambda_\cem{^{15}N} + \lambda_\cem{^{17}O} +\lambda_\cem{^{33}S}  	\\
	\eta 	&= 	\lambda_\cem{^{18}O} + \lambda_\cem{^{34}S}\\ 
	\gamma	&= 	\lambda_\cem{^{36}S}.
\end{align*}

The usefulness of approximation by a product of independent Poisson lies in two important properties, as summarised in the following lemmas.

\begin{lemma}\label{sum of independent Poissons lemma}
	Suppose we have a collection of $m$ independent Poisson-distributed random variables, $X_i \sim \mathrm{Poiss}(\kappa_i)$. Then $X_1 + \dots + X_m \sim \mathrm{Poiss}(\kappa_1 + \dots + \kappa_m)$. 
\end{lemma}  

\begin{lemma}\label{Poisson conditional on sum of Poissons}
	Suppose we have a collection of $m$ independent Poisson-distributed random variables, $X_i \sim \mathrm{Poiss}(\kappa_i)$. Then $X_1, \dots, X_m$ given that $X_1 + \dots + X_m = K$ is multinomially distributed,

$$ 
	\Big(X_1, \dots, X_m | X_1 + \dots + X_m = K \Big) 
	\sim 
	\mathrm{Multi}\Big( \frac{\kappa_1}{\sigma}, \dots, \frac{\kappa_m}{\sigma}; K \Big), 
$$
	where $\sigma = \sum_{i = 1}^m \kappa_i$.	
\end{lemma}
Both lemmas are proved in \cite{Kingman1993PoissonProcesses}. Lemma \ref{sum of independent Poissons lemma} shows how to simplify calculations for a Diophantine equations with all parameters set to one. Lemma \ref{Poisson conditional on sum of Poissons} describes the law resulting from conditioning independent Poisson variables by such an expression. 

Suppose that we concentrated on molecules composed entirely of elements that can have only one additional neutron, e.g. \smallMolecule. By Lemma \ref{Poisson conditional on sum of Poissons} we get:

\begin{result}\label{Multinomial Result}
 	For \smallMolecule, let $\tilde{\mu} := \lambda_\cem{^{13}C} + \lambda_\cem{^{2}H} + \lambda_\cem{^{15}N}$. Then
 	$$\QK = \mathrm{Multi}\left(
 		\frac{\lambda_\cem{^{13}C}}{\tilde{\mu}}, 
 		\frac{\lambda_\cem{^{2}H}}{\tilde{\mu}}, 
 		\frac{\lambda_\cem{^{15}N}}{\tilde{\mu}}; K \right).$$
\end{result}

\begin{proof}
	The corresponding Diophantine equation is $\cem{^{13}C} + \cem{^2H} + \cem{^{15}N} = K$.
\end{proof}

It is valuable to see, how Lemma \ref{Poisson conditional on sum of Poissons} generalizes while conditioning on a more complex Diophantine equation. Observe, that we can rewrite the definition of $LFS_K$ emphasizing the {\it equatransneutronic grouping}, i.e. glueing together counts of configurations with the same numbers of extra neutrons,
\begin{equation*}
	LFS_K = \Big\{ \underbrace{\cem{^{13}C} + \cem{^2H} + \cem{^{15}N} + \cem{^{17}O} + \cem{^{33}S}}_{ G_1 } + \,2 \times \underbrace{( \cem{^{18}O} + \cem{^{34}S} )}_{ G_2 } + \,4 \times \underbrace{\cem{^{36}S}}_{ G_4 } = K \Big\},	
\end{equation*}
so that in light of Lemma \ref{sum of independent Poissons lemma}, $\QQ(A)$ can be calculated in an easier way: 
$$\QQ( LFS_K ) = \sum_{k_1 + 2 k_2 + 4 k_4 = K} \prob( G_1 = k_1, G_2 = k_2, G_4 = k_4 ),$$
where $G_1 \sim \mathrm{Poiss}( \mu )$, $G_2 \sim \mathrm{Poiss}( \eta )$, and $G_4 \sim \mathrm{Poiss}( \gamma )$ are mutually independent. In light of \cite{Olson2009Calculations}, $G_i$ is equal to the total number of atoms bearing exactly $i$ additional neutrons.  

To calculate $\QK$ it remains to divide expression \eqref{QK Nominator} by $\QQ( LFS_K )$. Subsequent multiplication of both the nominator and the denominator of that result by $\frac{\mu^{k_1}}{k_1 !} \frac{\eta^{k_2}}{k_2 !} \frac{\gamma^{k_4}}{k_4 !}$ gives us an even more clear image of situation.

% Note also, that 
% \begin{align*}
% 	x 	& = \cem{c_1} + \cem{h_1} + \cem{n_1} + \cem{o_1} + \cem{s_1}, \\	  	
% 	y 	& = \cem{o_2} + \cem{s_2}, 	\\
% 	z 	& = \cem{s_4}.
% \end{align*}
% In \cite{Olson2009Calculations} they are encoded by $k_1, k_2$, and $k_4$; also, $d_{G_i} = i$. Then it is true that

\begin{result}\label{Fine structure law}
	The approximative \emph{fine structure law} with $K$ additional neutrons for \molecule\, is equal to 
	{\small
		\begin{equation*}
			\mathrm{Multi} \left(
				\frac{ \lambda_\cem{^{13}C} }{ \mu }, 
				\frac{ \lambda_\cem{^{2} H} }{ \mu }, 
				\frac{ \lambda_\cem{^{15}N} }{ \mu },
				\frac{ \lambda_\cem{^{17}O} }{ \mu }, 
				\frac{ \lambda_\cem{^{33}S} }{ \mu }; 
				k_1
			\right ) \otimes
			\mathrm{Multi} \left(
				\frac{ \lambda_\cem{^{18}O} }{ \eta },
				\frac{ \lambda_\cem{^{34}S} }{ \eta }; 
				k_2	
			\right) \otimes 
			\mathbb{L}( k_1, k_2, k_4 ),
		\end{equation*}
	}
	where 
	\begin{equation}\label{simple lucky law}
		\mathbb{L}( k_1, k_2, k_4 ) = 
		\frac{ \frac{ \mu^{k_1} }{ k_1! } \frac{ \eta^{k_2}}{ k_2! } \frac{ \gamma^{k_4} }{ k_4! } }{ 
			\underset{ k_1' + 2 k_2' + 4 k_4' = K}{\sum} 
				\frac{ \mu^{k_1'} }{ k_1'! } 
				\frac{ \eta^{k_2'}}{ k_2'! } 
				\frac{ \gamma^{k_4'}}{ k_4'! }
		}.
	\end{equation}
\end{result}

Otherwise stated, the approximative distribution is a mixture of independent multinomial distributions weighted by the $\mathbb{L}$ distribution, which, for lack of name, we shall call the {\it lucky law}. Under the Poisson approximation, the {\it lucky law} is the resulting law on the {\it equatransneutronic configurations}. General expression is to be found in the \textbf{Appendix}.

As pointed out in \cite{Olson2009Calculations}, it is of interest to calculate also the masses of the 
{\it equatransneutronic groups}. With Result \ref{Fine structure law}, we can provide extremely tractable approximations thereof. 
\begin{result}\label{Equatransneutronic Masses result}
	The approximative mass of a \emph{ transneutronic group} $(k_1, k_2, k_4)$ for compound \molecule\, is equal to
	{\small
		\begin{equation}\label{Equatransneutronic masses eq}
		\begin{gathered}
			\frac{k_1}{\mu}
			\Big( 
				{\Delta M}_\cem{^{13}C} \lambda_\cem{^{13}C} 	+ 
				{\Delta M}_\cem{^{2}H} 	\lambda_\cem{^{2}H} 	+ 
				{\Delta M}_\cem{^{15}N} \lambda_\cem{^{15}N} 	+ 
				{\Delta M}_\cem{^{18}O} \lambda_\cem{^{18}O} 	+  
				{\Delta M}_\cem{^{33}S} \lambda_\cem{^{33}S} 
			\Big)\\
			+\frac{k_2}{\eta}
			\Big( 
				{\Delta M}_\cem{^{18}O} \lambda_\cem{^{18}O} 	+ 
				{\Delta M}_\cem{^{34}S} \lambda_\cem{^{34}S} 
			\Big) + 
			\frac{k_4}{\gamma} {\Delta M}_\cem{^{36}S} \lambda_\cem{^{36}S} + 
			\textsc{Mono}_{ \cem{c}, \cem{h}, \cem{n}, \cem{o}, \cem{s} }, 
		\end{gathered}		
		\end{equation}}
	where $\Delta M$ stands for mass difference between a given isotope and the lightest isotope for that element, and \textsc{Mono} is the compound's monoisotopic mass.   
\end{result}
\begin{proof}
	It follows from the expression for multinomial law's mean, see \cite{Roos1999OnTheRateOfMultivariatePoissonConvergence}.
\end{proof}

Finally, note that other moments of the {\it equatransneutronic groupings} are readily obtained with the use of the multinomial moment generating function. 
%!TEX root = ../DeGaulle.tex
\section{Algorithms}

Result \ref{Fine structure law} opens up a new way to do calculations: by using the approximation one reduces the complexity of Problem \ref{Problem of finding LFS_K configurations.} to that of studying $\mathbb{L}$. Usually {\it Lucky Law} is less dimensional, and so the reduction is significant. In proteomics, one easily establishes the set all possible configurations $S$ in a double {\it for loop} and calculates their probabilites. In general, the problem could be approached using a tailored MCMC algorithm defined on the space of Linear Diophantine Equations.   
	
Having enumerated the {\it lucky} configurations, we order them by descending probability and select the critical $L\%$-set $S^*$. For each configuration $(k_1, k_2, k_4)$ in $S^*$ one can then independently find the $M\%$ and $B\%$-critical sets of the underlying multinomial distributions. We achieve this by controlled {\it breadth first search}: the configurations of the multinomial distribution can be thought of vertices $V$ of an underlying graph, $G = (V,E)$. Two configurations $\bm{v}, \bm{w} \in V$ define an edge $(\bm{v}, \bm{w}) \in E$ if and only if $\exists_{i \not= j} v_i = w_i + 1$ and $v_j = w_j - 1$. One then starts the algoritm in the vicinity of the mode of current $\mathrm{Multi}\left( p_1, \dots, p_w; n \right)$: as proxy, we use the point with coordinates equal to the floor of $n p_i + 1$. More elaborate set of candidates can be used, see \cite{Gall2003DeterminationOfTheModesOfMultinomial}. One then enlists all the neighbours of the initial node and puts the on a {\it max-priority queue}, see \cite{Cormen2001IntroductionToAlgorithms}. One then recursively looks at neighbours of the top-priority configuration, checks their probabiliy and enqueues them. In the same time, using a hash-table, one must store information on the visited configurations to avoid multiple visits to the same node. Observe that in case of molecules containing elements with only one isotope, e.g. \smallMolecule, this step alone would suffice to solve the problem, as showed in Result \ref{Multinomial Result}.

Having obtained the critical sets we calculate their exterior product and obtain a set of valid configurations in $LFS_K$. We calculate then their true probability under $\MK$ and their mass. Finally, we merge all obtained solutions.

We call the above algorithm {\sc DeFine}. A prototype of it has been implemented in \textbf{R}.	Fig. \ref{figure: Coverage} in  shows how well the prototype manages in solving Problem \ref{Problem of finding LFS_K configurations.}. Observe also, that the {\it for loop} can be carried out in parallel.

\begin{algorithm}\caption{\textsc{DeFine}}
\begin{algorithmic}[1]\label{DeFine Parallely}
	\State	\textbf{require:} \molecule, $K, L, M, B$
	\State 	Establish $\lambda_\cem{^{13}C}, \lambda_\cem{^{2} H}, \lambda_\cem{^{15}N}, \lambda_\cem{^{17}O}, \lambda_\cem{^{33}S}, \mu, \gamma $
	\State 	Find  		$S = \{ (k_1, k_2, k_4) : k_1 + 2 k_2 + 4 k_4 = K \}$.
	\State 	Find 		$P = \{ \prob(\bm{k}) : \bm{k} \in S \}$
	\State 	Order $S$ using $P$ and select the top $L \%$. Call result $S^*$.
	

	\FORALL{ $\bm{k} \in S^*$}
		\State $\mathfrak{M}$ := Critical $M\%$ set of $\mathrm{Multi} \left(
				\frac{ \lambda_\cem{^{13}C} }{ \mu }, 
				\frac{ \lambda_\cem{^{2} H} }{ \mu }, 
				\frac{ \lambda_\cem{^{15}N} }{ \mu },
				\frac{ \lambda_\cem{^{17}O} }{ \mu }, 
				\frac{ \lambda_\cem{^{33}S} }{ \mu }; 
				k_1
			\right )$

		\State $\mathfrak{B}$ \,:= Critical $B\%$ set of 
		$\mathrm{Multi} \left(
			\frac{ \lambda_\cem{^{18}O} }{ \eta },
			\frac{ \lambda_\cem{^{34}S} }{ \eta }; 
			k_2	
		\right)$		

		% \State $\mathfrak{C} \,\,\,:= \mathfrak{M} \otimes \mathfrak{B} \otimes \{ k_4 \}$

		\State 	Partial Result := 
			$\Bigg\{ 
				\Big( 
					\mathbb{M}( \bm{x}, \bm{y}, z ), 
					M( \bm{x}, \bm{y}, z )
				\Big) : 
				\bm{x} \in \mathfrak{M}, 
				\bm{y} \in \mathfrak{B}, 
				z = k_4  
			\Bigg\}$
	\ENDFORALL

	\State 	Result := $\bigcup $ Partial Results.
\end{algorithmic}	
\end{algorithm}
%!TEX root = ../DeGaulle.tex
\section{Conclusions}

	In the present paper an original approach to doing calculations on different levels of isotopic fine structure aggregation hierarchy was proposed. To our best knowledge, it is the first use of Poisson approximation for algorithmic purposes,   resulting already in two elegant algorithms, \textsc{DeFine} and \textsc{DeFiner}, for efficient exploration of the state space of possible isotopic configurations.  

	\textsc{DeFine} presents a minimalistic, yet extremely efficient way to calculate the approximate probabilities of {\it equatransneutronic} clusters. \textsc{DeFiner} presents a simple, yet certainly suboptimal way of handing Problem \ref{Problem of finding LFS_K configurations.}; however, more efficient algorithms can easily come into being by more careful considerations on the structure of approximative distribution $\QK$. 

	Figure \ref{figure: hierarchy} presents a detailed view of the hierarchical approach we take. The left pane contains the  aggregated isotopic distribution of \testAvergine, an $100$-avergine, obtained with the {\sc BRAIN} algorithm \cite{Dittwald2013BRAIN}. The lower panel zooms into the region of the highest aggregated peak. This peak is then disaggregated into {\it equatransneutronic} groupings. Finally, one notices many small black peaks corresponding to the finest structure obtainable. It is by clustering and statistical centroiding of these peaks that one obtains all the others. 

\begin{figure}[htbp]
 \centering
 \includegraphics[width=\textwidth]{./img/hierarchyHorizontal}
 \caption{ Peaks in the left pane are probabilities of different $LFS_K$ groups, $K = 0,\dots,13$. In the right pane masses of configurations in $LFS_6$ are plotted: it zooms the region around the tallest peak in the left pane, which is also plotted there for reference. By appropriately aggregating \textsc{DeFiner}'s results, i.e. small black peaks, we calculate the {\it equatransneutronic} precise, non-approximated probabilities, in blue. We compare them with \textsc{DeFine}'s results obtained {\it via} the Poisson approximation, in green. There are no apparent differences between them. }
 \label{figure: hierarchy}
\end{figure}


	The potential applications of our results are numerous. Above all, the fine structure models can find application in automatic top-down peptide identification procedures by establishing more detailed fingerprints thereof and possibly boosting the ability to differentiate between similar compounds. Differences in the fine structure with $K^*$ s.t. $\mathbb{M}_{K^*}(LSF_{K^*}) = \max_K \MK(LSF_K)$ could be particularly informative.


	As another application, one can ask how to set up an optimal binning procedure. Simply, with a critical set of configurations $A$, s.t. $\MK(A) \approx 95\%$, how should these configurations be glued together to match real data from a mass spectrometer. This way, one could measure the machine's resolution without a need to refer to somewhat underdefined notions of {\it p percent valley} and {\it peak width}, see \cite{Eidhammer2008ComputationalMethodsInMassSpectrometry}. 


	Observe that in this article we do not comment on the quality of the approximations in use. The reason behind it is that to our best knowledge no-one has ever carried out a thorough statistical research comparing which of these distributions is better suited for modelling the actual data. From the theoretical perspective, it seems plausible to adopt the most simple model of the isotopic fine structure probability, $\MM$, as developed in \cite{Kienitz1961MassSpectrometry}. However, with $\QQ$ at hand, and many data sets at disposal, one could verify whether such hypothesis holds. To our best knowledge, up to this moment only comparisons between theoretical distributions were carried out \cite{Valkenborg2007UsingPoisson}. We are of opinion that only through comparisons explicitly based on empirical data should one decide on the quality of the two models.



	% Moreover, at least in case of {\it Time of Flight} analyzers, there is an additional advantage of studying the {\it localised fine structure}, for their resolution is a function of the mass of analyte, see \cite{Eidhammer2008ComputationalMethodsInMassSpectrometry}. It is more difficult to differentiate correctly between molecules with similar masses, when both of them are big. 


	% However, we judge that all such algorithms could share the idea of using, in one way or another, the approach developed in {\sc Define}: namely, start by choosing a configuration presumed to be in vicinity of the mode of $\MK$, and proceed by a controled {\it breadth first search} until either a certain number of configurations is reached, or they already gathered ones already have enough of probability upon them.

	% A different problem to those mentioned before could be solved in this way to, namely:  

	% \begin{Problem}\label{Big Problem}
	% 	Find a small set $C$ among all possible configurations, s.t. $\mathbb{M}(C) \approx 1$.
	% \end{Problem}

	% The candidate for the biggest peak would by then be the product of modes of each multinomial model in \eqref{product of multinomials}.

	% This problem has been more efficiently be solved by different approaches, e.g. by the use of Fourier transform methods, see \todo{Find Rockwood's publication.}. 



	% Models solving Problem \ref{Big Problem} would have to add some sort of binning procedure with bin width being a function of mass, that not being straightforward to model. Thanks to the localisation in the mass to charge domain, while studying $LFS_K$ we simply neglect that sort of problem.


%!TEX root = ../DeGaulle.tex
\subsubsection*{Acknowledgments.}

We would like to thank Piotr Dittwald for thourough introduction to the problem of fine isotopic structure and many productive brawls over proper definitions. We also thank Dirk Valkenborg for pointing out the existence of the concept of equatransneutronic isotopes. Finally, a huge thanks goes to prof. Alan Rockwood, for the time he has spent teaching me about his approach to tackle the problem of isotope calculations.
  \bibliographystyle{./bst/splncs03.bst}
  \bibliography{./bib/spectrometry,./bib/probability,./bib/topology,./bib/generalMath,./bib/computerSciences}

  % no \newpage when using \include
%!TEX root = ../DeGaulle.tex
\section*{Tables}

\begin{table}[ht]
	\centering
	\caption{Basic Information on Stable Isotopes, as found in \cite{Rosman1997IsotopicCompositions}.}\label{basic info on isotopes table}
	\begin{tabular}{lccll}
		\toprule
Element 	& Isotope 		& Extra Neutrons& Mass [Da] & Probability 	\\
		\midrule
\multirow{2}{*}{Carbon}  	
			& \ce{^{12}C} 	& 0 			& 12 		& 0.9893 		\\	
  			& \ce{^{13}C} 	& 1 			& 13.0033 	& 0.0107 		\\	
  		\cmidrule(r){1-5}
\multirow{2}{*}{Hydrogen}  	
			& \ce{^1H} 		& 0 			& 1.0078 	& 0.999885 		\\	
  			& \ce{^2H} 		& 1 			& 2.0141 	& 0.000115 		\\	
  		\cmidrule(r){1-5}	
\multirow{2}{*}{Nitrogen}  	
			& \ce{^{14}N} 	& 0 			& 14.0031 	& 0.99632 		\\	
  			& \ce{^{15}N}	& 1 			& 15.0001 	& 0.00368 		\\	  	  
  		\cmidrule(r){1-5}	
\multirow{3}{*}{Oxygen}  	
			& \ce{^{16}O} 	& 0 			& 15.9949 	& 0.99757 		\\	
  			& \ce{^{17}O}	& 1 			& 16.9991 	& 0.00038 		\\	  	  	
  			& \ce{^{18}O}	& 2 			& 17.9992 	& 0.00205 		\\	  	  
  		\cmidrule(r){1-5}	
\multirow{4}{*}{Sulfur}  	
			& \ce{^{32}S} 	& 0 			& 31.9721 	& 0.9493 		\\	
  			& \ce{^{33}S}	& 1 			& 32.9714 	& 0.0076 		\\	  	  
  			& \ce{^{34}S}	& 2 			& 33.9679 	& 0.0429 		\\
  			& \ce{^{36}S}	& 4 			& 35.9671 	& 0.0002 		\\		
		\bottomrule
	\end{tabular}
\end{table}

%!TEX root = ../DeGaulle.tex
\section*{Figures}


\begin{figure}[htbp]
 \centering
 \includegraphics[width=.8\textwidth]{./img/DeFineCoverage}
 \caption{Coverage obtained using {\sc DeFinest} algorithm. The image on the bottom zooms into the upper reaches of the top picture. Both show the coverage of distribution original distribution $\MK$ for $K \in \{50, 100, 150, 200, 250, 300 \}$ for several chemical compounds. The bigger the compound (empirical formulas in the legend) the bigger the squares and the more intense the colour. Observe that for lighter compounds the results do not seem promising: we attribute this to the overall quality of conditional distributions $\MK$. Simply, all the multinomial distribution in \eqref{product of multinomials} are unimodal and for larger $K$ the solutions to Diophantine equation \eqref{LFS_K} do not encompass the region next to the mode, where the distribution is centered. For the reasons exposed in \textbf{Conclusions}, it is impractical to look at these distribution in the first place.}
 \label{figure: Coverage}
\end{figure}


%!TEX root = ../DeGaulle.tex
\section*{Appendix}

\subsection*{Proof of Lemma \ref{conditional convergence lemma}}

We want to prove that if $\mu^{[n]} \rightharpoonup \mu$ and $\mu^{[n]}(A), \mu(A) > 0$, then also $\mu^{[n]}_A \rightharpoonup \mu_A$. We do this under the assumption that both $\mu^{[n]}$ and $\mu$ are discrete measures on probability space $E$. 

By the {\it Portmanteau Lemma}, see \cite{Kallenberg2002FoundationsOfModernProbability}, $\mu^{[n]} \rightharpoonup \mu$ implies that for any set $A$ with boundry $\partial A$ subject to $\mu( \partial A) = 0$, one should observe 

\begin{equation}\label{convergence in probability on good sets}
	\lim_{n \to \infty} \mu^{[n]}(A) = \mu(A).
\end{equation}


The notion of boundry requires the notion of topology: thus, we decide on the discrete topology, which is natural in this context \footnote{For appropriate topological notions consult \cite{Dugundji1966Topology}.}. In this topology however, $\partial A = \emptyset$, for it is a set theoretical difference of the closure and the interior, both of which are equal to $A$. Hence, $\mu( \partial A) = 0$. Thus, \eqref{convergence in probability on good sets} always holds.

{\it Ex definitione}, $\mu^{[n]} \rightharpoonup \mu$ means, that for any bounded function $f:E\to\mathbb{R}$ one observes
\begin{equation}\label{weak convergence definition}
	\int f \mathrm{d} \mu^{[n]} \underset{n \to \infty}{\xrightarrow{\hspace*{1cm}}} \int f \mathrm{d}\mu\,.
\end{equation}

A simple calculation using both \eqref{convergence in probability on good sets} and \eqref{weak convergence definition} completes the proof:

\begin{equation*}
	\int f \mathrm{d} \mu^{[n]}_A =  \frac{ \int f \mathrm{d} \mu^{[n]} }{ \mu^{[n]}(A) } \underset{n \to \infty}{\xrightarrow{\hspace*{1cm}}} \frac{ \int f \mathrm{d} \mu }{ \mu(A) } = \int f \mathrm{d} \mu\,.
\end{equation*}

\subsection*{General form of the \emph{Lucky Law}}

If the compound contains elements with their {\it additional neutron acceptances} in set $I = \{ 1, 2, 4\}$, formula \eqref{simple lucky law} generalizes to 
\begin{equation*}
	\mathbb{L}( \bm{k} ) = 
	\frac{ 
		\prod_{i \in I} \frac{ \mu_i^{k_i} }{ {k_i}! } 
	}{ 
		\underset{ \{ \bm{k}^* :  \sum_{i \in I} i k_i^*  = K \} }{\sum} 	
		\prod_{i \in I} \frac{ \mu_i^{k_i^*} }{ {k_i^*}! }	
	},
\end{equation*}
where $\bm{k}$ is an ordered tuple indexed by $I$. Nature poses a natural limit on the complexity of the {\it lucky law}, as at most  $\# I \leq 10$\todo{Ascertain that asking Frederik.}.


\subsection*{Obtaining $M\%$ critical sets of the Multinomial Distribution}

We achieve this by controlled {\it breadth first search}: the configurations of the multinomial distribution can be thought of vertices $V$ of an underlying graph, $G = (V,E)$. Two configurations $\bm{v}, \bm{w} \in V$ define an edge $(\bm{v}, \bm{w}) \in E$ if and only if $\exists_{i \not= j} v_i = w_i + 1$ and $v_j = w_j - 1$. One then starts the algoritm in the vicinity of the mode of current $\mathrm{Multi}\left( p_1, \dots, p_w; n \right)$: as proxy, we use the point with coordinates equal to the floor of $n p_i + 1$. More elaborate set of candidates can be used, see \cite{Gall2003DeterminationOfTheModesOfMultinomial}. One then enlists all the neighbours of the initial node and puts the on a {\it max-priority queue}, see \cite{Cormen2001IntroductionToAlgorithms}. One then recursively looks at neighbours of the top-priority configuration, checks their probabiliy and enqueues them. In the same time, using a hash-table, one must store information on the visited configurations to avoid multiple visits to the same node. Observe that in case of molecules containing elements with only one isotope, e.g. \smallMolecule, this step alone would suffice to solve the problem, as showed in Result \ref{Multinomial Result}.


\end{document}
