%!TEX root = ../DeGaulle.tex
\section{Introduction}

Recent advances in mass spectrometric technology allow for a more and more elaborate application in biology. It is being recognised that more precise information can be retrieved even from larger chemical compounds. More resolved spectra already now help in the identification of complex mixtures of biomolecules, such as proteins and peptides, nucleic acids, and drugs; see \cite{Milandanovic2012OnTheUtilityOfIsotopicFineStructure}. 

It is well known that part of their complexity stems from the existence of stable isotopes. It is because of them that a given analyte is represented as a series of peaks, rather than just one corresponding to its monoisotopic mass. Depending on the machine, the isotopic structure can be resolved at different levels of accuracy. This provides a rationale for development of efficient algorithms that calculate their theoretical counterparts.

In this paper we consider three basic levels of aggregation of the isotopic structure,  corresponding to three distinct levels of theoretical resolution: the most coarse clumps together peaks with the same additional nucleon count, cf. \cite{Kienitz1961MassSpectrometry}, the finer one distinguishes between the {\it equatransneutronic groupings}, see \cite{Olson2009Calculations}, while the finest one represents completely resolved isotopic configurations, see \cite{Rockwood1995Ultrahighspeed}. The theoretical underpinnings of how to mathematically model the impact of isotopes are already well established, see \cite{Valkenborg2012Isotopic}, and the probability of a given exact fine configuration can be obtained using the product of multinomial distributions. However, with the growth of molecule one observes a general rise in the complexity in the problem of enumerating all fine isotopic configurations. To bypass this problem, different simplifications were proposed, amounting to different ways of binning configurations together explicitly \cite{Claesen2012Efficient} or by hiding them under the guise of Fourier Transform \cite{Rockwood1995Relationship}. 

Here we propose two refinements over the aggregate model, as used in \cite{Claesen2012Efficient}. Both of them use the concept of the {\it localised fine structure}, which corresponds to isotopic configurations clustered together into only one peak under the aggregate model. One of the devised algorithms extremely efficiently disaggregates that peak into {\it equatransneutronic groupings}; the other one fully resolves the isotopic pattern. Both of these algorithms are based on elegant Poisson approximations to the generally acknowledged multinomial model. To our best knowledge this type of approximation have not yet been used for algorithmic purposes. It has been used however in the context of proteomic and peptide research: in \cite{Breen2000AutomaticPeak} it was used for high throughput protein identification and then it was revalidated in \cite{Valkenborg2007UsingPoisson} for peptides. 