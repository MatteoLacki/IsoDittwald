%!TEX root = ../DeGaulle.tex
\section*{Appendix}

\subsection*{Proof of Lemma \ref{conditional convergence lemma}}

We want to prove that if $\mu^{[n]} \rightharpoonup \mu$ and $\mu^{[n]}(A), \mu(A) > 0$, then also $\mu^{[n]}_A \rightharpoonup \mu_A$. We do this under the assumption that both $\mu^{[n]}$ and $\mu$ are discrete measures on probability space $E$. 

By the {\it Portmanteau Lemma}, see \cite{Kallenberg2002FoundationsOfModernProbability}, $\mu^{[n]} \rightharpoonup \mu$ implies that for any set $A$ with boundry $\partial A$ subject to $\mu( \partial A) = 0$, one should observe 

\begin{equation}\label{convergence in probability on good sets}
	\lim_{n \to \infty} \mu^{[n]}(A) = \mu(A).
\end{equation}


The notion of boundry requires the notion of topology: thus, we decide on the discrete topology, which is natural in this context \footnote{For appropriate topological notions consult \cite{Dugundji1966Topology}.}. In this topology however, $\partial A = \emptyset$, for it is a set theoretical difference of the closure and the interior, both of which are equal to $A$. Hence, $\mu( \partial A) = 0$. Thus, \eqref{convergence in probability on good sets} always holds.

{\it Ex definitione}, $\mu^{[n]} \rightharpoonup \mu$ means, that for any bounded function $f:E\to\mathbb{R}$ one observes
\begin{equation}\label{weak convergence definition}
	\int f \mathrm{d} \mu^{[n]} \underset{n \to \infty}{\xrightarrow{\hspace*{1cm}}} \int f \mathrm{d}\mu\,.
\end{equation}

A simple calculation using both \eqref{convergence in probability on good sets} and \eqref{weak convergence definition} completes the proof:

\begin{equation*}
	\int f \mathrm{d} \mu^{[n]}_A =  \frac{ \int f \mathrm{d} \mu^{[n]} }{ \mu^{[n]}(A) } \underset{n \to \infty}{\xrightarrow{\hspace*{1cm}}} \frac{ \int f \mathrm{d} \mu }{ \mu(A) } = \int f \mathrm{d} \mu\,.
\end{equation*}

\subsection*{General form of the \emph{Lucky Law}}

If the compound contains elements with their {\it additional neutron acceptances} in set $I = \{ 1, 2, 4\}$, formula \eqref{simple lucky law} generalizes to 
\begin{equation*}
	\mathbb{L}( \bm{k} ) = 
	\frac{ 
		\prod_{i \in I} \frac{ \mu_i^{k_i} }{ {k_i}! } 
	}{ 
		\underset{ \{ \bm{k}^* :  \sum_{i \in I} i k_i^*  = K \} }{\sum} 	
		\prod_{i \in I} \frac{ \mu_i^{k_i^*} }{ {k_i^*}! }	
	},
\end{equation*}
where $\bm{k}$ is an ordered tuple indexed by $I$. Nature poses a natural limit on the complexity of the {\it lucky law}, as at most  $\# I \leq 10$\todo{Ascertain that asking Frederik.}.