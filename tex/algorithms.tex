%!TEX root = ../DeGaulle.tex
\section{Algorithms}

Result \ref{Fine structure law} opens up a new way to do calculations: by using the approximation one reduces the complexity of Problem \ref{Problem of finding LFS_K configurations.} to that of studying $\mathbb{L}$. Usually {\it Lucky Law} is less dimensional, and so the reduction is significant. In proteomics, one easily establishes the set all possible configurations $S$ in a double {\it for loop} and calculates their probabilites. In general, the problem could be approached using a tailored MCMC algorithm defined on the space of Linear Diophantine Equations.   
	
Having enumerated the {\it lucky} configurations, we order them by descending probability and select the critical $L\%$-set $S^*$. For each configuration $(k_1, k_2, k_4)$ in $S^*$ one can then independently find the $M\%$ and $B\%$-critical sets of the underlying multinomial distributions. This can be done in many ways, see \textbf{Appendix}.


Having obtained the critical sets we calculate their exterior product and obtain a set of valid configurations in $LFS_K$. We calculate then their true probability under $\MK$ and their mass, $M$. The mass of a configuration is obtained trivially.  Finally, we merge all obtained solutions.

We call the above algorithm {\sc DeFine}. A prototype of it has been implemented in \textbf{R}.	Fig. \ref{figure: Coverage} in  shows how well the prototype manages in solving Problem \ref{Problem of finding LFS_K configurations.}. Observe also, that the {\it for loop} can be carried out in parallel.

Say that the algorithm resulted in set $A$ of configurations. One can measure \textsc{DeFiner}'s perfomance simply by calculating the overall $\textsc{Coverage} := \MK(A)$; the higher it is, the better we get in solving Problem \ref{Problem of finding LFS_K configurations.}.



\begin{algorithm}\caption{\textsc{DeFiner}}
\begin{algorithmic}\label{DeFine Parallely}
	\State	\textbf{Input:} \molecule, $K, L, M, B$
	\State 	\textbf{Output:} List of pairs (Probability, Mass ), Coverage.
	\State 	Establish $\lambda_\cem{^{13}C}, \lambda_\cem{^{2} H}, \lambda_\cem{^{15}N}, \lambda_\cem{^{17}O}, \lambda_\cem{^{33}S}, \mu, \gamma $
	\State 	Find  		$S = \{ (k_1, k_2, k_4) : k_1 + 2 k_2 + 4 k_4 = K \}$.
	\State 	Find 		$P = \{ \prob(\bm{k}) : \bm{k} \in S \}$
	\State 	Order $S$ using $P$ and select the top $L \%$. Call result $S^*$.
	

	\FORALL{ $\bm{k} \in S^*$}
		\State $\mathfrak{M}$ := Critical $M\%$ set of $\mathrm{Multi} \left(
				\frac{ \lambda_\cem{^{13}C} }{ \mu }, 
				\frac{ \lambda_\cem{^{2} H} }{ \mu }, 
				\frac{ \lambda_\cem{^{15}N} }{ \mu },
				\frac{ \lambda_\cem{^{17}O} }{ \mu }, 
				\frac{ \lambda_\cem{^{33}S} }{ \mu }; 
				k_1
			\right )$

		\State $\mathfrak{B}$ \,:= Critical $B\%$ set of 
		$\mathrm{Multi} \left(
			\frac{ \lambda_\cem{^{18}O} }{ \eta },
			\frac{ \lambda_\cem{^{34}S} }{ \eta }; 
			k_2	
		\right)$		

		% \State $\mathfrak{C} \,\,\,:= \mathfrak{M} \otimes \mathfrak{B} \otimes \{ k_4 \}$

		\State 	Partial Result := {\small
		\begin{equation*}
				\Bigg\{ 
						\Big( 
							\MK ( \{\bm{x}, \bm{y}, z \}), 
							M( \bm{x}, \bm{y}, z )
						\Big) : \bm{x} = (\cem{c_1}, \cem{h_1}, \cem{n_1}, \cem{o_1}, \cem{s_1} ) \in \mathfrak{M},\quad \bm{y} = (\cem{o_2}, \cem{s_2}) \in \mathfrak{B},\quad z = \cem{s_4}  \Bigg\}			
		\end{equation*}}
		% \State 	Partial Result := 
		% 	$\Bigg\{ 
		% 		\Big( 
		% 			\MK ( \{\bm{x}, \bm{y}, z \}), 
		% 			M( \bm{x}, \bm{y}, z )
		% 		\Big) : 
		% 	$
		% \State\quad$\bm{x} = (\cem{c_1}, \cem{h_1}, \cem{n_1}, \cem{o_1}, \cem{s_1} ) \in \mathfrak{M},\quad \bm{y} = (\cem{o_2}, \cem{s_2}) \in \mathfrak{B},\quad z = \cem{s_4}  \Bigg\}$
	\ENDFORALL

	\State 	Result := $\bigcup $ Partial Results.
	\State 	Find {\sc Coverage}.
\end{algorithmic}	
\end{algorithm}