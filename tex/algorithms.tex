%!TEX root = ../DeGaulle.tex
\section{Algorithms}

Results \ref{Fine structure law} and \ref{Equatransneutronic Masses result} open up a new way to do calculations: using the approximation one reduces the complexity of Problem \ref{Problem of finding LFS_K configurations.} to that of studying $\mathbb{L}$. The {\it lucky law} is usually defined on a less dimensional space than $\MK$ and that significantly reduces the  computational effort. In proteomics, the state space of $\mathbb{L}$ can be thought to be two-dimensional, making it possible to establish the mass and probability of every {\it equatransneutronic grouping} in a {\it double for loop}. This approach is described as Algorithm \ref{DeFine code}, code-named \textsc{DeFine}. In general, exploration of $\mathbb{L}$ could be achieved by a tailored MCMC algorithm.   

\begin{algorithm}\caption{\textsc{DeFine}}\label{DeFine code}
\begin{algorithmic}\label{DeFine Parallely}
	\State	\textbf{Input:}\,\,\,\,\, \molecule, $K$
	\State 	\textbf{Output:} A triplet of arrays with configurations, their probability, and mass.
	\State 	Establish $\lambda_\cem{^{13}C}, \lambda_\cem{^{2} H}, \lambda_\cem{^{15}N}, \lambda_\cem{^{17}O}, \lambda_\cem{^{33}S}, \mu, \gamma $, and all mass differences $\Delta M$.
	\State 	Find   $S = \{ (k_1, k_2, k_4) : k_1 + 2 k_2 + 4 k_4 = K \}$.
	\FORALL{ $\bm{k} \in S$}
		\State 	$\mathrm{Lucky}(\bm{k}) := \mathbb{L}\Big( \{ \bm{k} \} \Big)$
		\State 	$M(\bm{k}) :=$ mass of configuration $\bm{k}$ obtained using Eq. \eqref{Equatransneutronic masses eq} 
	\ENDFORALL
	\State 	Return $\{ S, \mathrm{Lucky}, M\}$.
\end{algorithmic}	
\end{algorithm}

\textsc{DeFine} can be used as a subroutine for \textsc{DeFiner}: an algorithm that provides the exact multinomial peaks. \textsc{DeFiner} works as follows. 

First, having obtained the {\it lucky} configurations, we order them in descending $\mathbb{L}$-probability and select the critical $L\%$-set $S_L$ to trim out the asymptotically negligible configurations. To show it is so, let us introduce some extra notation
\begin{equation}
	\bm{g}_1 := (\cem{c_1}, \cem{h_1}, \cem{n_1}, \cem{o_1}, \cem{s_1} ), \quad
	\bm{g}_2 := (\cem{o_2}, \cem{s_2}), \quad
	g_4 := \cem{s_4}, \quad 
	\bm{g} := ( \bm{g}_1, \bm{g}_2, g_4)
\end{equation}
We think of $\bm{g}_1$ and $\bm{g}_2$ as of configurations of the multinomial laws described in Result \ref{Fine structure law}. Observe that entries of $\bm{g}_i$ sum to $k_i$. By Result \ref{Fine structure law}, note that $\QK \big( S_L \big) \leq \QK \big( S \big) = \mathbb{L}\big( \{ \bm{k} \} \big)$, where $\bm{k} := (k_1, k_2, k_4)$, the probability of a peak in a given {\it equatransneutronic grouping} being smaller than the probability of all the peaks gathered in it. Therefore, assymptotically all the $\bm{g}$ configurations in $S_L$ have a small $\QK$-probability, and we can decide whether to neglect them using only the information contained in $\mathbb{L}\big( \{ \bm{k} \} \big)$.

Subsequently, for each configuration $\bm{k}$ in $S_L$, one independently identifies critical $M\%$-set $\mathfrak{M}$ and critical $B\%$-set $\mathfrak{B}$ of the two underlying multinomial distributions. This can be achieved in many ways, see \textbf{Appendix} for our approach. With these sets at hand we calculate their exterior product and obtain a set of valid configurations from $LFS_K$. We then find their true $\MK$-probability and their mass $M$ using Eq. \eqref{Equatransneutronic masses eq}. We make use of \textsc{BRAIN} \cite{Dittwald2013BRAIN} software to get $\MM(LFS_K)$ needed to calculate $\MK$. Finally, we merge all obtained solutions.

Say that the algorithm resulted in set $A$ of configurations. One can measure \textsc{DeFiner}'s performance simply by calculating the overall $\textsc{Coverage} := \MK(A)$; the higher it is, the better we are in solving Problem \ref{Problem of finding LFS_K configurations.}.

A prototype of \textsc{DeFiner} has been implemented in \textbf{R}. Its pseudo code is described as Algorithm \ref{DeFiner code}. Observe also, that the {\it for loop} can be carried out in parallel. Fig. \ref{figure: Coverage} shows how well the prototype manages in solving Problem \ref{Problem of finding LFS_K configurations.}. 




\begin{algorithm}\caption{\textsc{DeFiner}}\label{DeFiner code}
\begin{algorithmic}
	\State	\textbf{Input:} \molecule, $K, L, M, B$
	\State 	\textbf{Output:} array of masses and probabilities, \textsc{Coverage}.
	\State 	Run \textsc{DeFine} and obtain $\{ S, \mathrm{Lucky}, M\}$.
	\State 	$S_L$:= top $L\%$ of configurations from $S$ ordered by their {\it lucky probabilities}, $\mathbb{L}(\{ \bm{k} \})$.
	

	\FORALL{ $\bm{k} \in S_L$}
		\State $\mathfrak{M}$ := Critical $M\%$ set of $\mathrm{Multi} \left(
				\frac{ \lambda_\cem{^{13}C} }{ \mu }, 
				\frac{ \lambda_\cem{^{2} H} }{ \mu }, 
				\frac{ \lambda_\cem{^{15}N} }{ \mu },
				\frac{ \lambda_\cem{^{17}O} }{ \mu }, 
				\frac{ \lambda_\cem{^{33}S} }{ \mu }; 
				k_1
			\right )$

		\State $\mathfrak{B}$ \,:= Critical $B\%$ set of 
		$\mathrm{Multi} \left(
			\frac{ \lambda_\cem{^{18}O} }{ \eta },
			\frac{ \lambda_\cem{^{34}S} }{ \eta }; 
			k_2	
		\right)$		

		\State $\mathfrak{R}_{\bm{k}}$ := $ 
			\Bigg\{ 
				\Big( 
					\MK( \bm{g} ), 
					M( \bm{g} )
				\Big) : \bm{g} = ( \bm{g}_1, \bm{g}_2, g_4 ) \in \mathfrak{M} \otimes \mathfrak{B} \otimes \{ k_4 \}  
			\Bigg\}$, cf. Eq. \eqref{Equatransneutronic masses eq} 
	\ENDFORALL

	\State 	Find {\sc Coverage}.
	\State 	Return $\big\{ \bigcup _{\bm{k}} \mathfrak{R}_{\bm{k}}, \textsc{Coverage} \big\}$.
\end{algorithmic}	
\end{algorithm}